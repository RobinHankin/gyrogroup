% -*- mode: noweb; noweb-default-code-mode: R-mode; -*-
\documentclass[nojss]{jss}
\usepackage{dsfont}
\usepackage{bbm}
\usepackage{amsfonts}
\usepackage{amsmath}
\usepackage{wasysym}
\usepackage{amssymb}
\usepackage{yfonts}

%%%%%%%%%%%%%%%%%%%%%%%%%%%%%%
%% declarations for jss.cls %%%%%%%%%%%%%%%%%%%%%%%%%%%%%%%%%%%%%%%%%%
%%%%%%%%%%%%%%%%%%%%%%%%%%%%%%


%% just as usual
\author{Robin K. S. Hankin\\Auckland University of Technology}
\title{Noncommutative and nonassociative three velocity in special relativity: introducing the \pkg{gyrogroup} package}


%\VignetteIndexEntry{The gyrogroup package}

%% for pretty printing and a nice hypersummary also set:
\Plainauthor{Robin K. S. Hankin}
\Plaintitle{The gyrogroup package}
\Shorttitle{The gyrogroup package}

%% an abstract and keywords
\Abstract{

Here I present the \pkg{gyrogroup} package for generalized
Bradley-Terry models and give examples from two competetive
situations: single scull rowing, and the competitive cooking game show
MasterChef Australia.  A number of natural statistical hypotheses may
be tested straightforwardly using the software.
}


\Keywords{Dirichlet distribution, hyperdirichlet distribution,
  combinatorics, \proglang{R}, multinomial distribution, constrained
optimization, Bradley-Terry}
\Plainkeywords{Dirichlet distribution, hyperdirichlet distribution,
  combinatorics, R, multinomial distribution, constrained
optimization, Bradley-Terry}
  
  
  
%% publication information
%% NOTE: This needs to filled out ONLY IF THE PAPER WAS ACCEPTED.
%% If it was not (yet) accepted, leave them commented.
%% \Volume{13}
%% \Issue{9}
%% \Month{September}
%% \Year{2004}
%% \Submitdate{2004-09-29}
%% \Acceptdate{2004-09-29}

%% The address of (at least) one author should be given
%% in the following format:
\Address{
  Robin K. S. Hankin\\
  Auckland University of Technology\\
  E-mail: \email{hankin.robin@gmail.com}
}
%% It is also possible to add a telephone and fax number
%% before the e-mail in the following format:
%% Telephone: +43/1/31336-5053
%% Fax: +43/1/31336-734

%% for those who use Sweave please include the following line (with % symbols):
%% need no \usepackage{Sweave.sty}

%% end of declarations %%%%%%%%%%%%%%%%%%%%%%%%%%%%%%%%%%%%%%%%%%%%%%%


\newcommand{\bu}{\mathbf u}
\newcommand{\bv}{\mathbf v}
\newcommand{\bw}{\mathbf w}
\newcommand{\bx}{\mathbf x}
\newcommand{\by}{\mathbf y}

\DeclareMathOperator{\gyr}{gyr}


\begin{document}
\section{Introduction}



\section{Introduction}

``The nonassociativity of Einstein's velocity addition is not widely
known''~\citep{ungar2006}.

In this short vignette I will introduce the \pkg{gyrogroup} package which gives
functionality for manipulating three-velocities in the context of their being a gyrogroup.

\section{The package in use}

\cite{ungar2006} shows that the velocity addition law is

\begin{equation}
\bu\oplus\bv=
\frac{1}{1+\bu\cdot\bv}
\left\{
\bu + \frac{\bv}{\gamma_\bu} + \frac{\gamma_\bu
\left(\bu\cdot\bv\right)\bu}{1+\gamma_\bu}
\right\}
\end{equation}
   
where~$\gamma_\bu=\left(1-\bu\cdot\bu\right)^{-1/2}$ and we are
assuming~$c=1$.  Ungar shows that, in general,
$\bu\oplus\bv\neq\bv\oplus\bu$
and~$(\bu\oplus\bv)\oplus\bw\neq\bu\oplus(\bv\oplus\bw)$.  He also
defines the binary operator~$\ominus$
as~$\bu\ominus\bv=\bu\oplus\left(-\bv\right)$ (and implicitly
defines~$\ominus\bu\oplus\bv$ to be~$\left(-\bu\right)\oplus\bv$).

If we have

\begin{equation}
\gyr\left[\bu,\bv\right]\bx=-\left(\bu\oplus\bv\right)\oplus\left(\bu\oplus\left(\bv\oplus\bx\right)\right)
\end{equation}

Then Ungar shows that 

\begin{eqnarray}
\gyr\left[\bu,\bv\right]\bx\cdot\gyr\left[\bu,\bv\right]x &=& \bx\cdot\by\\
\gyr\left[\bu,\bv\right]\left(\bx\oplus\by\right) &=& \gyr\left[\bu,\bv\right]\bx\oplus\gyr\left[\bu,\bv\right]\by\\
\left(\gyr\left[\bu,\bv\right]\right)^{-1} &=& \left(\gyr\left[\bv,\bu\right]\right)\\
\bu\oplus\bv &=& \gyr\left[\bu,\bv\right]\left(\bv\oplus\bu\right)\\
\bu\oplus\left(\bv\oplus\bw\right) &=&\left(\bu\oplus\bv\right)\oplus\gyr\left[\bu,\bv\right]\bw\\
\left(\bu\oplus\bv\right)\oplus\bw &=&\bu\oplus\left(\bv\oplus\gyr\left[\bv,\bu\right]\bw\right)
\end{eqnarray}

Consider the following R session:

\begin{Schunk}
\begin{Sinput}
> library(gyrogroup)
> u <- as.3vel(c(-0.7,+0.2,-0.3))
> v <- as.3vel(c(+0.3,+0.3,+0.4))
> w <- as.3vel(c(+0.4,-0.3,+0.5))
\end{Sinput}
\end{Schunk}

Here we have three-vectors \code{u},  and \code{v}.


\section{Conclusions}

Several generalization of Bradley-Terry strengths are appropriate to
describe competitive situations in which order statistics are
sufficient.

The package is used to calculate maximum likelihood estimates for
generalized Bradley-Terry strengths in two competitive situations:
Olympic rowing, and \emph{MasterChef Australia}.  The estimates for
the competitors' strengths are plausible; and several meaningful
statistical hypotheses are assessed quantitatively.

\bibliography{gyrogroup}
\end{document}
 
