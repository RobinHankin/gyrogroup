%%%%%%%%%%%%%%%%%%%%%%%%%%%%%%%%%%%%%%%%%%%%%%%%%%%%%%%%%%%%%%%%%%%%%%%%%%%%
%% Author template for INFORMS Journal on Computing (ijoc)
%% Mirko Janc, Ph.D., INFORMS, mirko.janc@informs.org
%% ver. 0.95, December 2010
%%%%%%%%%%%%%%%%%%%%%%%%%%%%%%%%%%%%%%%%%%%%%%%%%%%%%%%%%%%%%%%%%%%%%%%%%%%%
%\documentclass[ijoc,blindrev]{informs3}
\documentclass[ijoc,nonblindrev]{informs3} % current default for manuscript submission

%%\OneAndAHalfSpacedXI
\OneAndAHalfSpacedXII % current default line spacing
%%\DoubleSpacedXII
%%\DoubleSpacedXI

% If hyperref is used, dvi-to-ps driver of choice must be declared as
%   an additional option to the \documentclass. For example
%\documentclass[dvips,ijoc]{informs3}      % if dvips is used
%\documentclass[dvipsone,ijoc]{informs3}   % if dvipsone is used, etc.

% Private macros here (check that there is no clash with the style)

% Natbib setup for author-year style
\usepackage{natbib}
 \bibpunct[, ]{(}{)}{,}{a}{}{,}%
 \def\bibfont{\small}%
 \def\bibsep{\smallskipamount}%
 \def\bibhang{24pt}%
 \def\newblock{\ }%
 \def\BIBand{and}%

%% Setup of theorem styles. Outcomment only one. 
%% Preferred default is the first option.
\TheoremsNumberedThrough     % Preferred (Theorem 1, Lemma 1, Theorem 2)
%\TheoremsNumberedByChapter  % (Theorem 1.1, Lema 1.1, Theorem 1.2)

%% Setup of the equation numbering system. Outcomment only one.
%% Preferred default is the first option.
\EquationsNumberedThrough    % Default: (1), (2), ...
%\EquationsNumberedBySection % (1.1), (1.2), ...

% In the reviewing and copyediting stage enter the manuscript number.
%\MANUSCRIPTNO{} % When the article is logged in and DOI assigned to it,
                 %   this manuscript number is no longer necessary

%%%%%%%%%%%%%%%%
\begin{document}
%%%%%%%%%%%%%%%%

% Outcomment only when entries are known. Otherwise leave as is and 
%   default values will be used.
%\setcounter{page}{1}
%\VOLUME{00}%
%\NO{0}%
%\MONTH{Xxxxx}% (month or a similar seasonal id)
%\YEAR{0000}% e.g., 2005
%\FIRSTPAGE{000}%
%\LASTPAGE{000}%
%\SHORTYEAR{00}% shortened year (two-digit)
%\ISSUE{0000} %
%\LONGFIRSTPAGE{0001} %
%\DOI{10.1287/xxxx.0000.0000}%

% Author's names for the running heads
% Sample depending on the number of authors;
% \RUNAUTHOR{Jones}
% \RUNAUTHOR{Jones and Wilson}
% \RUNAUTHOR{Jones, Miller, and Wilson}
% \RUNAUTHOR{Jones et al.} % for four or more authors
% Enter authors following the given pattern:
\RUNAUTHOR{Hankin}

% Title or shortened title suitable for running heads. Sample:
% \RUNTITLE{Bundling Information Goods of Decreasing Value}
% Enter the (shortened) title:
\RUNTITLE{Special relativity in R}

% Full title. Sample:
% \TITLE{Bundling Information Goods of Decreasing Value}
% Enter the full title:
\TITLE{Special relativity in R: introducing the lorentz package}

% Block of authors and their affiliations starts here:
% NOTE: Authors with same affiliation, if the order of authors allows, 
%   should be entered in ONE field, separated by a comma. 
%   \EMAIL field can be repeated if more than one author
\ARTICLEAUTHORS{%
\AUTHOR{Robin K. S. Hankin}
\AFF{AUT, \EMAIL{hankin.robin@gmail.com}, \URL{}}
% Enter all authors
} % end of the block

\ABSTRACT{%

Here I present the {\tt lorentz} package for working with relativistic
physics.  The package includes functionality for Lorentz tnransforms
and Einsteinian three-velocity addition, which is noncommutative and
nonassociative.  The package is used to search for a distributive law
for relativistic three-velocities.

}%

% Sample 
%\KEYWORDS{deterministic inventory theory; infinite linear programming duality; 
%  existence of optimal policies; semi-Markov decision process; cyclic schedule}

% Fill in data. If unknown, outcomment the field
\KEYWORDS{Lorentz transform, Lorentz group, Lorentz law, Lorentz
  velocity addition, special relativity, relativistic physics,
  Einstein velocity addition, Wigner rotation, gyrogroup,
  gyromorphism, gyrocommutative, gyroassociative, four velocity,
  three-velocity, nonassociative, noncommutative}
\HISTORY{}

\maketitle
%%%%%%%%%%%%%%%%%%%%%%%%%%%%%%%%%%%%%%%%%%%%%%%%%%%%%%%%%%%%%%%%%%%%%%

% Samples of sectioning (and labeling) in IJOC
% NOTE: (1) \section and \subsection do NOT end with a period
%       (2) \subsubsection and lower need end punctuation
%       (3) capitalization is as shown (title style).
%
%\section{Introduction.}\label{intro} %%1.
%\subsection{Duality and the Classical EOQ Problem.}\label{class-EOQ} %% 1.1.
%\subsection{Outline.}\label{outline1} %% 1.2.
%\subsubsection{Cyclic Schedules for the General Deterministic SMDP.}
%  \label{cyclic-schedules} %% 1.2.1
%\section{Problem Description.}\label{problemdescription} %% 2.

% Text of your paper here




\newcommand{\bu}{\mathbf u}
\newcommand{\bv}{\mathbf v}
\newcommand{\bw}{\mathbf w}
\newcommand{\bx}{\mathbf x}
\newcommand{\by}{\mathbf y}

\newcommand{\gyr}[2]{\operatorname{gyr}\left[{\mathbf #1},{\mathbf #2}\right]}


In special relativity, the Lorentz transforms supercede their
classical equivalent, the Galilean transforms~\citep{goldstein1980}.
Lorentz transforms operate on four-vectors such as the four-velocity
or four-potential and are usually operationalised as multiplication by
a $4\times 4$ matrix.  A Lorentz transform takes the components of an
arbitrary four-vector as observed in one coordinate system and returns
the components observed in another system which is moving at constant
velocity with respect to the first.

There are a few existing software tools for working with Lorentz
transforms, mostly developed in an educational context.  Early work
would include that of \citet{horwitz1992} who describe {\tt relLab}, a
system for building a range of {\em gendanken} experiments in an
interactive graphical environment.  The author asserts that it runs on
``any Macintosh computer with one megabyte of RAM or more'' but it is
not clear whether the software is still available.  More modern
contributions would include the {\tt OpenRelativity}
toolkit~\citep{sherin2016} which simulates the effects of special
relativity in the {\tt Unity} game engine.

The {\tt lorentz} package provides {\tt R}-centric functionality for
Lorentz transforms.  It deals with formal Lorentz boosts, converts
between three-velocities and four-velocities, and provides
computational support for the gyrogroup structure of relativistic
three-velocity addition.



\section{Lorentz transforms: active and passive}

Passive transforms are the usual type of transforms taught and used in
relativity.  However, sometimes active transforms are needed and it is
easy to confuse the two.  Here I will discuss passive and then active
transforms, and illustrate both in a computational context.

\subsection*{Passive transforms}

\newcommand{\vvec}[2]{\begin{pmatrix}#1 \\ #2\end{pmatrix}}
\newcommand{\twomat}[4]{\begin{pmatrix} #1 & #2 \\ #3 &
    #4\end{pmatrix}}

Consider the following canonical Lorentz transform in which we have
motion in the $x$-direction at speed $v>0$; the motion is from left to
right.  We consider only the first two components of four-vectors, the
$y$- and $z$- components being trivial.  A typical physical
interpretation is that I am at rest, and a colleague is in his
spaceship moving at speed $v$ past me; and we are wondering what
vectors which I measure in my own rest frame look like to him.  The
(passive) Lorentz transform is:

\begin{equation*}
\twomat{\gamma}{-\gamma v}{-\gamma v}{\gamma}
\end{equation*}

And the canonical example of that would be:

\begin{equation*}
\twomat{\gamma}{-\gamma v}{-\gamma
  v}{\gamma}\vvec{1}{0}=\vvec{\gamma}{-\gamma v}
\end{equation*}

where the vectors are four velocities (recall that $\vvec{1}{0}$ is
the four-velocity of an object at rest).  Operationally, I measure the
four-velocity of an object to be $\vvec{1}{0}$, and he measures the
same object as having a four-velocity of $\vvec{\gamma}{-\gamma v}$.
So I see the object at rest, and he sees it as moving at speed $-v$;
that is, he sees it moving to the left (it moves to the left because
he is moving to the right relative to me).  The package makes
computations easy.  Suppose $v=0.6c$ in the $x$-direction.

\begin{verbatim}

> u <- as.3vel(c(0.6,0,0)) # coerce to a three-velocity u

       x y z 
[1,] 0.6 0 0

> as.4vel(u) # four-velocity is better for calculations

        t    x y z
[1,] 1.25 0.75 0 0

\end{verbatim}

For more general work one usually uses boosts, $4\times 4$
transformation matrices:

\begin{verbatim}
> boost(u)
      t     x y z
   1.25 -0.75 0 0 
x -0.75  1.25 0 0 
y  0.00  0.00 1 0 
z  0.00  0.00 0 1

\end{verbatim}

\section{Successive Lorentz transforms}

Coordinate transformation is effected by standard matrix
multiplication; thus composition of two Lorentz transforms is also
ordinary matrix multiplication:

\begin{verbatim}

> u <- as.3vel(c(0.3,-0.4,+0.8))
> v <- as.3vel(c(0.4,+0.2,-0.1))
> L <- boost(u) %*% boost(v)
> L

          t          x          y          z
t  3.256577 -2.2327055  0.5419596 -2.0800479
x -1.437147  1.6996791 -0.0237489  0.4194255
y  1.091131 -0.7581795  1.1190282 -0.6029155
z -2.519789  1.5878378 -0.2023170  2.1879612

\end{verbatim}

But observe that the resulting transform is not a pure boost, as the
spatial components are not symmetrical.  We may decompose the matrix
product $L$ into a pure translation composed with an orthogonal
matrix, which represents a coordinate rotation.  The package provides
extensive R idiom for effecting this composition.

\section{Three-velocities}

In contrast to four-velocities, three-velocities do not form a group
under composition as the velocity addition law is not
associative~\citep{ungar2006}.  Instead, three-velocity composition
has an algebraic structure known as a {\em gyrogroup} (this
observation was the original motivation for the package).
\citeauthor{ungar2006} shows that the velocity addition law for
three-velocities is

\begin{equation}
\bu\oplus\bv= \frac{1}{1+\bu\cdot\bv} \left\{ \bu +
\frac{\bv}{\gamma_\bu} + \frac{\gamma_\bu
  \left(\bu\cdot\bv\right)\bu}{1+\gamma_\bu} \right\}
\end{equation}
   
where~$\gamma_\bu=\left(1-\bu\cdot\bu\right)^{-1/2}$ and we are
assuming~$c=1$.  \citeauthor{ungar2006} goes on to show that, in
general, $\bu\oplus\bv\neq\bv\oplus\bu$
and~$(\bu\oplus\bv)\oplus\bw\neq\bu\oplus(\bv\oplus\bw)$.  He also
defines the binary operator~$\ominus$
as~$\bu\ominus\bv=\bu\oplus\left(-\bv\right)$, and implicitly
defines~$\ominus\bu\oplus\bv$ to be~$\left(-\bu\right)\oplus\bv$.  If
we have

\begin{equation}
\gyr{u}{v}\bx=-\left(\bu\oplus\bv\right)\oplus\left(\bu\oplus\left(\bv\oplus\bx\right)\right)
\end{equation}

then

\begin{eqnarray}
\bu\oplus\bv &=&
\gyr{u}{v}\left(\bv\oplus\bu\right)\label{noncom}\\ \gyr{u}{v}\bx\cdot\gyr{u}{v}\by
&=& \bx\cdot\by\label{doteq}\\ \gyr{u}{v}\left(\bx\oplus\by\right) &=&
\gyr{u}{v}\bx\oplus\gyr{u}{v}\by\\ \left(\gyr{u}{v}\right)^{-1} &=&
\left(\gyr{v}{u}\right)\label{gyrinv}\\ \bu\oplus\left(\bv\oplus\bw\right)
&=&\left(\bu\oplus\bv\right)\oplus\gyr{u}{v}\bw\label{nonass1}\\ \left(\bu\oplus\bv\right)\oplus\bw
&=&\bu\oplus\left(\bv\oplus\gyr{v}{u}\bw\right)\label{nonass2}
\end{eqnarray}

Consider the following R session:

\begin{verbatim}
> u <- as.3vel(c(-0.7,+0.2,-0.3))
> v <- as.3vel(c(+0.3,+0.3,+0.4))
> w <- as.3vel(c(+0.1,+0.3,+0.8))
> x <- as.3vel(c(-0.2,-0.1,-0.9))

\end{verbatim}

Here we have three-vectors {\tt u} etc.  We can see that {\tt u} and
{\tt v} do not commute:

\begin{verbatim}
> u+v

          x     y        z
[1,] -0.545 0.482 -0.00454

> v+u

          x     y     z
[1,] -0.429 0.572 0.132

\end{verbatim}

(the results differ).  We can use equation~\ref{noncom}

\begin{verbatim}

> (u+v)-gyr(u,v,v+u)

            x         y        z 
[1,] 1.77e-16 -7.08e-16 1.23e-16

\end{verbatim}

showing agreement to within numerical error.  It is also possible to
use the functional idiom in which we define {\tt f()} to be the
map~$\bx\mapsto\gyr{u}{v}\bx$.

\subsection{Associativity}

Three velocity addition is not associative:

\begin{verbatim}
> (u+v)+w

          x     y     z
[1,] -0.465 0.655 0.501

> u+(v+w)

          x     y     z 
[1,] -0.549 0.667 0.416

\end{verbatim}

But we can use equations~\ref{nonass1} and~\ref{nonass2}:

\begin{verbatim}
> (u+(v+w)) - ((u+v)+gyr(u,v,w))

            x         y         z
[1,] 6.92e-16 -1.38e-15 -6.92e-16

> ((u+v)+w) - (u+(v+gyr(v,u,w)))

     x y        z
[1,] 0 0 5.35e-16

\end{verbatim}

\subsection{Visualization of noncommutativity and nonassociativity of three-velocities}

Consider the following three-velocities:

\begin{verbatim}
> u <- as.3vel(c(0.4,0,0))
> v <- seq(as.3vel(c(0.4,-0.2,0)), as.3vel(c(-0.3,0.9,0)),len=20)
> w <- as.3vel(c(0.8,-0.4,0))

\end{verbatim}

Objects~$\bv$ and $\bw$ are single three-velocities, and object $\bv$
 is a vector of three velocities.  We can see the noncommutativity of
 three velocity addition in figures~\ref{comfail1} and~\ref{comfail2},
 and the nonassociativity in figure~\ref{assfail}.

\begin{figure}[htbp]
  \begin{center}
\includegraphics{lorentz-comfail1_fig}
\caption{Failure\label{comfail1} of the commutative law for velocity
  composition in special relativity.  The arrows show successive
  velocity boosts of $+\bu$ (purple), $+\bv$ (black), $-\bu$ (red),
  and~$-\bv$ (blue) for $\bu,\bv$ as defined above.  Velocity $\bu$ is
  constant, while $\bv$ takes a sequence of values.  If velocity
  addition is commutative, the four boosts form a closed
  quadrilateral; the thick arrows show a case where the boosts almost
  close and the boosts nearly form a parallelogram.  The blue dots
  show the final velocity after four successive boosts; the distance
  of the blue dot from the origin measures the combined velocity,
  equal to zero in the classical limit of low speeds.  The discrepancy
  becomes larger and larger for the faster elements of the sequence
  $\bv$}
  \end{center}
\end{figure}

\begin{figure}[htbp]
  \begin{center}
\includegraphics{lorentz-comfail2_fig}
\caption{Another view of the failure of the commutative
  law\label{comfail2} in special relativity.  The black arrows show
  velocity boosts of $\bu$ and the blue arrows show velocity boosts of
  $\bv$, with $\bu,\bv$ as defined above; $\bu$ is constant while
  $\bv$ takes a sequence of values.  If velocity addition is
  commutative, then $\bu+\bv=\bv+\bu$ and the two paths end at the
  same point: the parallelogram is closed.  The red lines show the
  difference between $\bu+\bv$ and $\bv+\bu$}
  \end{center}
\end{figure}

\begin{figure}[htbp]
  \begin{center}
\includegraphics{lorentz-assfail_fig}
\caption{Failure of the associative law \label{assfail} for velocity
  composition in special relativity.  The arrows show successive
  boosts of $\bu$ followed by $\bv+\bw$ (black lines), and $\bu+\bv$
  followed by $\bw$ (blue lines), for $\bu$, $\bv$, $\bw$ as defined
  above; $\bu$ and $\bw$ are constant while $\bv$ takes a sequence of
  values. The mismatch between $\bu+\left(\bv+\bw\right)$ and
  $\left(\bu+\bv\right)+\bw$ is shown in red}
  \end{center}
\end{figure}


\section{Distributivity in special relativity}

Three-velocities are not distributive in special relativiity.  That
is, given three-velocities $\bu$, $\bv$ and a scalar $r$, we have that
$r(\bu+\bv)\neq r\bu+r\bv$ in general.  Ungar (2006) goes on to
suggest that a generalized distributive law would be an interesting
and instructive object, if it existed.  The {\tt lorentz} package was
written to leverage the structure of R in the finding of a
distributive law.  By way of illustration, possible expressions for 
$r(\bu+\bv)$ might include

\begin{itemize}
\item $r\bu  + (r\bv  - r\operatorname{gyr}[r\bu, \bv,\bu+\bv])$
\item $r\bu  + (r\bv  - r\operatorname{gyr}[r\bu, \bv,\bv+\bu])$
\item $r\bv  + (r\bu  - r\operatorname{gyr}[s\bu, \bv,\bu+\bv])$
\item $(r\bu + r\bv)  + r\operatorname{gyr}[r\bu, \bv,\bu]$
\item $(r\bu + r\bv)  + r\operatorname{gyr}[r\bu, \bv,\bv]$
\item $(r\bu + r\bv)  + r\operatorname{gyr}[r\bu, \bv,\bv]$
\item $(r\bu + r\bv)  + r\operatorname{gyr}[r\bu,-\bv,\bv]$
\item $(r\bu + r\bv)  +   \operatorname{gyr}[r\bu,-\bv,\bv]$
\item $(r\bu + r\bv)  + s\operatorname{gyr}[r\bu,-\bv,\bv]$
\end{itemize}

Here, $r$ is a scalar, $s=1/r$, and $\bu$, $\bv$ are three-velocities.
Note the variety of orders (three-velocity addition is not
commutative), different bracketing (three-velocity addition is not
associative), and use of $r$ or $s=1/r$ in different places in the
formula.  The {\tt lorentz package} facilitates a systematic sweep
through such plausible distributive laws, and the latest revision
searches a total of 688128 possibilities, unfortunately without
success.

\section{Conclusions}

The {\tt lorentz} package furnishes some functionality for
manipulating four-vectors and three-velocities in the context of
special relativity.  The R idiom is relatively natural and the package
has been used to illustrate different features of relativistic
kinematics.

 



% Appendix here
% Options are (1) APPENDIX (with or without general title) or 
%             (2) APPENDICES (if it has more than one unrelated sections)
% Outcomment the appropriate case if necessary
%
% \begin{APPENDIX}{<Title of the Appendix>}
% \end{APPENDIX}
%
%   or 
%
% \begin{APPENDICES}
% \section{<Title of Section A>}
% \section{<Title of Section B>}
% etc
% \end{APPENDICES}


% References here (outcomment the appropriate case) 

% CASE 1: BiBTeX used to constantly update the references 
%   (while the paper is being written).
\bibliographystyle{informs2014} % outcomment this and next line in Case 1
\bibliography{lorentz} % if more than one, comma separated

% CASE 2: BiBTeX used to generate mypaper.bbl (to be further fine tuned)
%\input{mypaper.bbl} % outcomment this line in Case 2

\end{document}


