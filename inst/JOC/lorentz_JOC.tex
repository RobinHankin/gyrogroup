%%%%%%%%%%%%%%%%%%%%%%%%%%%%%%%%%%%%%%%%%%%%%%%%%%%%%%%%%%%%%%%%%%%%%%%%%%%%
%% Author template for INFORMS Journal on Computing (ijoc)
%% Mirko Janc, Ph.D., INFORMS, mirko.janc@informs.org
%% ver. 0.95, December 2010
%%%%%%%%%%%%%%%%%%%%%%%%%%%%%%%%%%%%%%%%%%%%%%%%%%%%%%%%%%%%%%%%%%%%%%%%%%%%
%\documentclass[ijoc,blindrev]{informs3}
\documentclass[ijoc,nonblindrev]{informs3} % current default for manuscript submission

%%\OneAndAHalfSpacedXI
\OneAndAHalfSpacedXII % current default line spacing
%%\DoubleSpacedXII
%%\DoubleSpacedXI

% If hyperref is used, dvi-to-ps driver of choice must be declared as
%   an additional option to the \documentclass. For example
%\documentclass[dvips,ijoc]{informs3}      % if dvips is used
%\documentclass[dvipsone,ijoc]{informs3}   % if dvipsone is used, etc.

% Private macros here (check that there is no clash with the style)

% Natbib setup for author-year style
\usepackage{natbib}
 \bibpunct[, ]{(}{)}{,}{a}{}{,}%
 \def\bibfont{\small}%
 \def\bibsep{\smallskipamount}%
 \def\bibhang{24pt}%
 \def\newblock{\ }%
 \def\BIBand{and}%

%% Setup of theorem styles. Outcomment only one. 
%% Preferred default is the first option.
\TheoremsNumberedThrough     % Preferred (Theorem 1, Lemma 1, Theorem 2)
%\TheoremsNumberedByChapter  % (Theorem 1.1, Lema 1.1, Theorem 1.2)

%% Setup of the equation numbering system. Outcomment only one.
%% Preferred default is the first option.
\EquationsNumberedThrough    % Default: (1), (2), ...
%\EquationsNumberedBySection % (1.1), (1.2), ...

% In the reviewing and copyediting stage enter the manuscript number.
%\MANUSCRIPTNO{} % When the article is logged in and DOI assigned to it,
                 %   this manuscript number is no longer necessary

%%%%%%%%%%%%%%%%
\begin{document}
%%%%%%%%%%%%%%%%

% Outcomment only when entries are known. Otherwise leave as is and 
%   default values will be used.
%\setcounter{page}{1}
%\VOLUME{00}%
%\NO{0}%
%\MONTH{Xxxxx}% (month or a similar seasonal id)
%\YEAR{0000}% e.g., 2005
%\FIRSTPAGE{000}%
%\LASTPAGE{000}%
%\SHORTYEAR{00}% shortened year (two-digit)
%\ISSUE{0000} %
%\LONGFIRSTPAGE{0001} %
%\DOI{10.1287/xxxx.0000.0000}%

% Author's names for the running heads
% Sample depending on the number of authors;
% \RUNAUTHOR{Jones}
% \RUNAUTHOR{Jones and Wilson}
% \RUNAUTHOR{Jones, Miller, and Wilson}
% \RUNAUTHOR{Jones et al.} % for four or more authors
% Enter authors following the given pattern:
\RUNAUTHOR{Hankin}

% Title or shortened title suitable for running heads. Sample:
% \RUNTITLE{Bundling Information Goods of Decreasing Value}
% Enter the (shortened) title:
\RUNTITLE{Special relativity in R}

% Full title. Sample:
% \TITLE{Bundling Information Goods of Decreasing Value}
% Enter the full title:
\TITLE{Special relativity in R: introducing the ``{\tt lorentz}'' package}

% Block of authors and their affiliations starts here:
% NOTE: Authors with same affiliation, if the order of authors allows, 
%   should be entered in ONE field, separated by a comma. 
%   \EMAIL field can be repeated if more than one author
\ARTICLEAUTHORS{%
\AUTHOR{Robin K. S. Hankin}
\AFF{Auckland University of Technology, \EMAIL{hankin.robin@gmail.com} \URL{}}
% Enter all authors
} % end of the block

\ABSTRACT{%

Here I present the {\tt lorentz} package for working with relativistic
physics in the R computer language.  The package includes
functionality for Lorentz transforms and Einsteinian three-velocity
addition, which is noncommutative and nonassociative.  The package is
used to search for a distributive law for relativistic
three-velocities.

}%

% Sample 
%\KEYWORDS{deterministic inventory theory; infinite linear programming duality; 
%  existence of optimal policies; semi-Markov decision process; cyclic schedule}

% Fill in data. If unknown, outcomment the field
\KEYWORDS{Lorentz transform, Lorentz group, Lorentz law, Lorentz
  velocity addition, special relativity, relativistic physics,
  Einstein velocity addition, Wigner rotation, gyrogroup,
  gyromorphism, gyrocommutative, gyroassociative, four velocity,
  three-velocity, nonassociative, noncommutative}
\HISTORY{}

\maketitle
%%%%%%%%%%%%%%%%%%%%%%%%%%%%%%%%%%%%%%%%%%%%%%%%%%%%%%%%%%%%%%%%%%%%%%

% Samples of sectioning (and labeling) in IJOC
% NOTE: (1) \section and \subsection do NOT end with a period
%       (2) \subsubsection and lower need end punctuation
%       (3) capitalization is as shown (title style).
%
%\section{Introduction.}\label{intro} %%1.
%\subsection{Duality and the Classical EOQ Problem.}\label{class-EOQ} %% 1.1.
%\subsection{Outline.}\label{outline1} %% 1.2.
%\subsubsection{Cyclic Schedules for the General Deterministic SMDP.}
%  \label{cyclic-schedules} %% 1.2.1
%\section{Problem Description.}\label{problemdescription} %% 2.

% Text of your paper here







\newcommand{\gyr}[2]{\operatorname{gyr}\left[{\mathbf #1},{\mathbf #2}\right]}


In special relativity, the Lorentz transforms supercede their
classical equivalent, the Galilean transforms~\citep{goldstein1980}.
Lorentz transforms operate on four-vectors such as the four-velocity
or four-potential and are usually operationalised as multiplication by
a $4\times 4$ matrix.  A Lorentz transform takes the components of an
arbitrary four-vector as observed in one coordinate system and returns
the components observed in another system which is moving at constant
velocity with respect to the first.

There are a few existing software tools for working with Lorentz
transforms, mostly developed in an educational context.  Early work
would include that of \citet{horwitz1992} who describe {\tt relLab}, a
system for building a range of {\em gendanken} experiments in an
interactive graphical environment.  The author asserts that it runs on
``any Macintosh computer with one megabyte of RAM or more'' but it is
not clear whether the software is still available.  More modern
contributions would include the {\tt OpenRelativity}
toolkit~\citep{sherin2016} which simulates the effects of special
relativity in the {\tt Unity} game engine.

The R computer language~\citep{rcore2020} is usually used for
statistical computation but the language is sufficiently flexible to
be applied to diverse numerical disciplines such as complex
analysis~\citep{hankin2006}, recreational
mathematics~\citep{hankin2005}, and knot theory~\citep{hankin2017}.

The {\tt lorentz} package provides R-centric functionality for
relativistic velocity addition.  It deals with formal Lorentz boosts,
converts between three-velocities and four-velocities, and provides
computational support for the gyrogroup structure of relativistic
three-velocity addition.  Package source code is available on github
at \url{https://github.com/RobinHankin/lorentz}, and CRAN, at
\url{https://CRAN.R-project.org/package=lorentz}.



\section{Lorentz transforms}

\newcommand{\vvec}[2]{\begin{pmatrix}#1 \\ #2\end{pmatrix}}
\newcommand{\twomat}[4]{\begin{pmatrix} #1 & #2 \\ #3 &
    #4\end{pmatrix}}


Here I will discuss (passive) Lorentz transforms in a computational
context.  Consider the following canonical Lorentz transform in which
we have motion in the $x$-direction at speed $v>0$; the motion is from
left to right.  A typical physical interpretation is that I am at
rest, and a colleague is in his spaceship moving at speed
$\BFu=\left(u,v,w\right)$ past me; we wonder what vectors which I
measure in my own rest frame look like to him.  The appropriate
coordinate transform would be:

\begin{equation}
  \begin{pmatrix}
\gamma &-\gamma u&-\gamma v&-\gamma w\\
    -\gamma u&1+(\gamma-1)n_xn_x&(\gamma-1)n_xn_y&(\gamma-1)n_xn_z\\
    -\gamma v&(\gamma-1)n_yn_x&1+(\gamma-1)n_yn_y&(\gamma-1)n_yn_z\\
    -\gamma w&(\gamma-1)n_zn_x&(\gamma-1)n_zn_y&1+(\gamma-1)n_zn_z
  \end{pmatrix}
\end{equation}

where $\gamma=\left(1-\BFu\cdot\BFu\right)^{-1/2}$ and $n_x,n_y,n_z$
are the direction cosines of $\BFu$.  The package makes this type of
computation easy and employs natural R idiom.  Suppose, for example,
that $\BFu=\left(0.1,0.6,-0.4\right)$:

\begin{verbatim}

> u <- as.3vel(c(0.1, 0.6, -0.4))
> u
       x   y    z
[1,] 0.1 0.6 -0.4

> as.4vel(u)    # four-velocity is more useful in physics
           t        x         y        z
[1,] 1.45865 0.145865 0.8751899 -0.58346

\end{verbatim}

For more general work one usually uses boosts, $4\times 4$
transformation matrices:

\begin{verbatim}
> boost(u) 
           t           x           y           z
t  1.4586499 -0.14586499 -0.87518995  0.58345997
x -0.1458650  1.00865377  0.05192263 -0.03461509
y -0.8751899  0.05192263  1.31153579 -0.20769053
z  0.5834600 -0.03461509 -0.20769053  1.13846035
> 

\end{verbatim}

\section{Successive Lorentz transforms}

Coordinate transformation is effected by standard matrix
multiplication; thus composition of two Lorentz transforms is also
ordinary matrix multiplication:

\begin{verbatim}

> u <- as.3vel(c(0.3,-0.4,+0.8))
> v <- as.3vel(c(0.4,+0.2,-0.1))
> L <- boost(u) %*% boost(v)
> L

          t          x          y          z
t  3.256577 -2.2327055  0.5419596 -2.0800479
x -1.437147  1.6996791 -0.0237489  0.4194255
y  1.091131 -0.7581795  1.1190282 -0.6029155
z -2.519789  1.5878378 -0.2023170  2.1879612

\end{verbatim}

But observe that the resulting transform is not a pure boost, as the
spatial components are not symmetrical\footnote{Identifying pure
  boosts with symmetric transformation matrices applies only if $c=1$.
  The package includes extensive capability for dealing with units in
  which $c\neq 1$, but the algebra becomes more complicated.}.  We may
decompose the matrix product $L$ into a pure translation composed with
an orthogonal matrix, which represents a coordinate rotation.  The
package provides extensive R idiom for effecting this composition.

\section{Three-velocities}

In contrast to four-velocities, three-velocities do not form a group
under composition as the velocity addition law is not
associative~\citep{ungar2006}.  Instead, three-velocity composition
has an algebraic structure known as a {\em gyrogroup} (this
observation was the original motivation for the package).
\citeauthor{ungar2006} shows that the velocity addition law for
three-velocities is

\begin{equation}
\BFu\oplus\BFv= \frac{1}{1+\BFu\cdot\BFv} \left\{ \BFu +
\frac{\BFv}{\gamma_\BFu} + \frac{\gamma_\BFu
  \left(\BFu\cdot\BFv\right)\BFu}{1+\gamma_\BFu} \right\}
\end{equation}
   
where~$\gamma_\BFu=\left(1-\BFu\cdot\BFu\right)^{-1/2}$ and we are
assuming~$c=1$.  \citeauthor{ungar2006} goes on to show that, in
general, $\BFu\oplus\BFv\neq\BFv\oplus\BFu$
and~$(\BFu\oplus\BFv)\oplus\BFw\neq\BFu\oplus(\BFv\oplus\BFw)$.  He also
defines the binary operator~$\ominus$
as~$\BFu\ominus\BFv=\BFu\oplus\left(-\BFv\right)$, and implicitly
defines~$\ominus\BFu\oplus\BFv$ to be~$\left(-\BFu\right)\oplus\BFv$.  If
we have

\begin{equation}
\gyr{u}{v}\BFx=-\left(\BFu\oplus\BFv\right)\oplus\left(\BFu\oplus\left(\BFv\oplus\BFx\right)\right)
\end{equation}

then

\begin{eqnarray}
\BFu\oplus\BFv &=&
\gyr{u}{v}\left(\BFv\oplus\BFu\right)\label{noncom}\\ \gyr{u}{v}\BFx\cdot\gyr{u}{v}\BFy
&=& \BFx\cdot\BFy\label{doteq}\\ \gyr{u}{v}\left(\BFx\oplus\BFy\right) &=&
\gyr{u}{v}\BFx\oplus\gyr{u}{v}\BFy\\ \left(\gyr{u}{v}\right)^{-1} &=&
\left(\gyr{v}{u}\right)\label{gyrinv}\\ \BFu\oplus\left(\BFv\oplus\BFw\right)
&=&\left(\BFu\oplus\BFv\right)\oplus\gyr{u}{v}\BFw\label{nonass1}\\ \left(\BFu\oplus\BFv\right)\oplus\BFw
&=&\BFu\oplus\left(\BFv\oplus\gyr{v}{u}\BFw\right)\label{nonass2}
\end{eqnarray}

Consider the following R session:

\begin{verbatim}
> u <- as.3vel(c(-0.7,+0.2,-0.3))
> v <- as.3vel(c(+0.3,+0.3,+0.4))
> w <- as.3vel(c(+0.1,+0.3,+0.8))
> x <- as.3vel(c(-0.2,-0.1,-0.9))

\end{verbatim}

Here we have three-vectors {\tt u} etc.  We can see that {\tt u} and
{\tt v} do not commute:

\begin{verbatim}
> u+v

          x     y        z
[1,] -0.545 0.482 -0.00454

> v+u

          x     y     z
[1,] -0.429 0.572 0.132

\end{verbatim}

(the results differ).  We can verify equation~\ref{noncom}:

\begin{verbatim}

> (u+v)-gyr(u,v,v+u)

            x         y        z 
[1,] 1.77e-16 -7.08e-16 1.23e-16

\end{verbatim}

showing agreement to within numerical error.  It is also possible to
use the functional idiom in which we define {\tt f()} to be the
map~$\BFx\mapsto\gyr{u}{v}\BFx$.

\subsection{Associativity}

Three-velocity addition is nonassociative, a phenomenon described as
``bizarre and counterintuitive'' by \cite{ungar1997}.  We can give a
clear numerical illustration of this using standard package idiom:

\begin{verbatim}
> (u+v)+w

          x     y     z
[1,] -0.465 0.655 0.501

> u+(v+w)

          x     y     z 
[1,] -0.549 0.667 0.416

\end{verbatim}

But we can use equations~\ref{nonass1} and~\ref{nonass2}:

\begin{verbatim}
> (u+(v+w)) - ((u+v)+gyr(u,v,w))

            x         y         z
[1,] 6.92e-16 -1.38e-15 -6.92e-16

> ((u+v)+w) - (u+(v+gyr(v,u,w)))

     x y        z
[1,] 0 0 5.35e-16

\end{verbatim}

which verify the consistency of the gyrogroup structure of
three-velocity additoion.


\subsection{Visualization of noncommutativity and nonassociativity of three-velocities}

Consider the following three-velocities:

\begin{verbatim}
> u <- as.3vel(c(0.4,0,0))
> v <- seq(as.3vel(c(0.4,-0.2,0)), as.3vel(c(-0.3,0.9,0)),len=20)
> w <- as.3vel(c(0.8,-0.4,0))

\end{verbatim}

Objects~$\BFv$ and $\BFw$ are single three-velocities, and object $\BFv$
 is a vector of three velocities.  We can see the noncommutativity of
 three velocity addition in figures~\ref{comfail1} and~\ref{comfail2},
 and the nonassociativity in figure~\ref{assfail}.

\begin{figure}[htbp]
  \begin{center}
\includegraphics{lorentz-comfail1_fig}
\caption{Failure\label{comfail1} of the commutative law for velocity
  composition in special relativity.  The arrows show successive
  velocity boosts of $+\BFu$ (purple), $+\BFv$ (black), $-\BFu$ (red),
  and~$-\BFv$ (blue) for $\BFu,\BFv$ as defined above.  Velocity $\BFu$ is
  constant, while $\BFv$ takes a sequence of values.  If velocity
  addition is commutative, the four boosts form a closed
  quadrilateral; the thick arrows show a case where the boosts almost
  close and the boosts nearly form a parallelogram.  The blue dots
  show the final velocity after four successive boosts; the distance
  of the blue dot from the origin measures the combined velocity,
  equal to zero in the classical limit of low speeds.  The discrepancy
  becomes larger and larger for the faster elements of the sequence
  $\BFv$}
  \end{center}
\end{figure}

\begin{figure}[htbp]
  \begin{center}
\includegraphics{lorentz-comfail2_fig}
\caption{Another view of the failure of the commutative
  law\label{comfail2} in special relativity.  The black arrows show
  velocity boosts of $\BFu$ and the blue arrows show velocity boosts of
  $\BFv$, with $\BFu,\BFv$ as defined above; $\BFu$ is constant while
  $\BFv$ takes a sequence of values.  If velocity addition is
  commutative, then $\BFu+\BFv=\BFv+\BFu$ and the two paths end at the
  same point: the parallelogram is closed.  The red lines show the
  difference between $\BFu+\BFv$ and $\BFv+\BFu$}
  \end{center}
\end{figure}

\begin{figure}[htbp]
  \begin{center}
\includegraphics{lorentz-assfail_fig}
\caption{Failure of the associative law \label{assfail} for velocity
  composition in special relativity.  The arrows show successive
  boosts of $\BFu$ followed by $\BFv+\BFw$ (black lines), and $\BFu+\BFv$
  followed by $\BFw$ (blue lines), for $\BFu$, $\BFv$, $\BFw$ as defined
  above; $\BFu$ and $\BFw$ are constant while $\BFv$ takes a sequence of
  values. The mismatch between $\BFu+\left(\BFv+\BFw\right)$ and
  $\left(\BFu+\BFv\right)+\BFw$ is shown in red}
  \end{center}
\end{figure}


\section{Distributivity in special relativity}

A relativistic three-velocity $\BFu$ may be ``multiplied'' by a scalar
$r$ with the law
$r\BFu=\sinh\left(r\sinh^{-1}\left|\left|\BFu\right|\right|\right)\BFu$,
thus for example $2\BFu=\BFu\oplus\BFu$~\citep{ungar1997}.
Three-velocities are not distributive in special relativity.  That is,
given three-velocities $\BFu$, $\BFv$ and a scalar $r$, we have that
$r(\BFu\oplus\BFv)\neq r\BFu\oplus r\BFv$ in general.
\cite{ungar1997} considers this and expresses the hope that one day a
gyrodistributive law expressing $r\left(\BFu\oplus\BFv\right)$ will be
discovered; and that, if it exists, is ``expected to be the standard
distributive law [i.e. $r\BFu\oplus r\BFv$] relaxed by Thomas gyration
in some unexpected way''.  The {\tt lorentz} package was written to
leverage the structure of R in the finding of a distributive law.  By
way of illustration, possible expressions for $r(\BFu\oplus\BFv)$ might
include

\begin{itemize}
\item $r\BFu  \oplus (r\BFv  \ominus \operatorname{gyr}[r\BFu, \BFv,\BFu\oplus\BFv])$
\item $r\BFu  \oplus (r\BFv  \ominus \operatorname{gyr}[r\BFu, \BFv,\BFv\oplus\BFu])$
\item $r\BFv  \oplus (r\BFu  \ominus r\operatorname{gyr}[s\BFu, \BFv,\BFu\oplus\BFv])$
\item $(r\BFu \oplus r\BFv)  \oplus r\operatorname{gyr}[r\BFu, \BFv,\BFu]$
\item $(r\BFu \oplus r\BFv)  \oplus r\operatorname{gyr}[r\BFu, \BFv,\BFv]$
\item $(r\BFu \oplus r\BFv)  \oplus \operatorname{gyr}[r\BFu, \BFv,\BFv]$
\item $(r\BFu \oplus r\BFv)  \oplus r\operatorname{gyr}[r\BFu,-\BFv,\BFv]$
\item $(r\BFu \oplus r\BFv)  \oplus   \operatorname{gyr}[r\BFu,-\BFv,\BFv]$
\item $(r\BFu \oplus r\BFv)  \oplus s\operatorname{gyr}[r\BFu,-\BFv,\BFv]$.
\end{itemize}

Here, $r$ is a scalar, $s=1/r$, and $\BFu$, $\BFv$ are three-velocities.
Note the variety of orders (three-velocity addition is not
commutative), different bracketing (three-velocity addition is not
associative), and use of $r$ or $s=1/r$ in different places in the
formula.  The {\tt lorentz package} facilitates a systematic sweep
through such plausible distributive laws, and the latest revision
searches a total of 688128 possibilities, unfortunately without
success.

\section{Conclusions}

The {\tt lorentz} package furnishes some functionality for
manipulating four-vectors and three-velocities in the context of
special relativity.  The R idiom is efficient and natural, and the
package illustrates different features of relativistic kinematics 
The package was used to carry out a systematic, but unsuccessful,
search for a distributive law.


% Appendix here
% Options are (1) APPENDIX (with or without general title) or 
%             (2) APPENDICES (if it has more than one unrelated sections)
% Outcomment the appropriate case if necessary
%
% \begin{APPENDIX}{<Title of the Appendix>}
% \end{APPENDIX}
%
%   or 
%
% \begin{APPENDICES}
% \section{<Title of Section A>}
% \section{<Title of Section B>}
% etc
% \end{APPENDICES}


% References here (outcomment the appropriate case) 

% CASE 1: BiBTeX used to constantly update the references 
%   (while the paper is being written).
\bibliographystyle{informs2014} % outcomment this and next line in Case 1
\bibliography{lorentz} % if more than one, comma separated

% CASE 2: BiBTeX used to generate mypaper.bbl (to be further fine tuned)
%\input{mypaper.bbl} % outcomment this line in Case 2

\end{document}


